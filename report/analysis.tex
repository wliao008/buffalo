\chapter{Analysis}

The project hypothesized that by using Buffalo, programmers can separate the cross-cutting concerns from their application quickly and easily. While the current implementation of Buffalo cannot be shoehorned into every situation, in many scenarios cross-cutting concerns can be decoupled and encapsulated into distinct unit of code as an aspect, represented as a .NET attribute type. This separation of concern is what really encourages the code re-usability. Buffalo facilitates this process by providing the necessary plumbing. 

While it is difficult to quantify the benefit of using Buffalo, it offers as a solid alternative to managing the cross-cutting problem. It enables aspect creations to capture and isolate problems that are spreading into different modules of the program. Because Buffalo makes use of the System.Attribute type that already existed on the .NET platform, developers will have the full power of the .NET framework at their disposal when creating the aspects. 

Getting started with Buffalo is also relatively easy compare to other frameworks that are configuration based, where an interface must map to a concrete implementation in order for code injection to work. Buffalo has no such configuration file requirement. Buffalo is 100\% managed code; developers already familiar with the .NET framework will have minimum learning curve.

Although the title of this report indicates that the framework is developed in C\#, by no mean is it limited to the C\# language. Since Buffalo works with the Common Intermediate Language, in practice Buffalo can be used with other .NET languages that compile to CIL, such as VB.NET. Preliminary tests show that Buffalo can also work on the Linux environment where the Mono C\# compiler is installed.

Generally speaking, when using Buffalo, it also makes the code a bit cleaner to read, in that it abstracts away the common code into aspects. Since aspects are centralized, maintainability is much easier. 

Buffalo can be useful in situation where cross-cutting concerns are present. One common case is exception logging, where an aspect can be developed to catch all the unhandled exception.

One can argue that since unhandled exceptions will bubble up the chain, a developer can simply catch them in the main method, and this would have achieved similar effect. That approach is limited in that it is very generic. When the main method catches an exception, it has no idea what the internal state of the failed method was. Using Buffalo, the internal state can be inspected by checking the MethodArgs object, giving developer a better sense at what caused the failure.

Besides exception handling, another common problem facing developers working on database is to ensure data atomicity. Imagine a customer who deposits money to a bank, the amount is saved successfully to the bank’s own record, but the customer’s account failed to get updated due to unforeseeable network error. Not only would the database end up with incomplete data, but the bank would have a very unhappy customer. Either all operations are successfully completed or nothing should have been written to the database.

Buffalo can be used to solve this problem by encapsulating transactional scope using the MethodBoundaryAspect. Complete example is available in Appendix A.

Developers that work with UI elements often run into problems updating the UI from a different thread. As UI elements can be updated only by thread that created them, this creates the cross-threading problem. Buffalo’s MethodAroundAspect can be used to get around this problem by marshaling the update back to the UI thread. For more detail please refer to Appendix A.

Buffalo allows developers to quickly create aspects to solve various problems. Besides the scenarios mentioned above, it can also be used for authorization, where security credential is validated before method execution. It can be used to create a caching mechanism to improve application performance. Buffalo has performed well in isolating cross-cutting concerns into single unit of code, which is easily maintained and modified.
