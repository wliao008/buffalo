\chapter{Introduction}

Object Oriented Programming (OOP) languages have given programmers a lot of freedom in expressing themselves in Object Oriented Design. However, they are still lacking in some areas when it comes to particular software design decision such as cross-cutting concern.

Cross-cutting concerns are aspects of the program that scatter throughout different areas of the code base. These aspects usually cannot be easily or cleanly separated from the rest of the program. Aspect Oriented Programming~\cite{aop} in essence is to find ways to solve the cross-cutting concern problems.

In this project, a framework called "Buffalo" is designed and implemented to solve this type of problem on the .NET platform. Buffalo makes use of the .NET attribute system to weave aspect code to any targeted methods. The design and rationale of the framework is discussed in section 2. The implementation detail is given in section 3.

The result indicates that by using Buffalo, developers can separate cross-cutting concerns from the core of the program for easy maintenance, and ultimately be more productive. The analysis is discussed in section 4.

The report concludes in section 5 with the current project status. A set of planned future works is also discussed and what is learned from working on this project.

Buffalo comprises around 1,200 lines of source codes. A user manual is included in Appendix A, which contains some examples on how the system works. Instructions are also included on how to integrate Buffalo with MS-Build.
