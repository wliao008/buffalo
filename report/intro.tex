\chapter{Introduction}

Object Oriented Programming (OOP) languages have given programmers much freedom in expressing themselves in Object Oriented Design. However, they are still lacking in some areas when it comes to particular software design decisions such as cross-cutting concern.

Cross-cutting concerns are aspects of the program that scatter throughout different areas of the code base. These aspects cannot usually be separated easily or cleanly from the rest of the program. The essence of Aspect Oriented Programming~\cite{aop} is to find ways to solve the cross-cutting concern problems.

In this project, a framework called "Buffalo" is designed and implemented to solve this type of problem on the .NET platform. Buffalo makes use of the .NET attribute system to weave aspect code to any targeted methods. The design and rationale of the framework is discussed in Chapter 2. The implementation detail is given in Chapter 3.

The results indicate that by using Buffalo, developers can separate cross-cutting concerns from the core of the program for easy maintenance and ultimately be more productive. The analysis is discussed in Chapter 4.

The report concludes in Chapter 5 with the current project status. A set of planned future works is also discussed as well as what was learned from working on this project.

Buffalo comprises around 1,200 lines of source code. A user manual is included in Appendix A, which contains examples on how the system works. Instructions are also included for how to integrate Buffalo with MS-Build.
