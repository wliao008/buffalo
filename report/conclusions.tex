\chapter{Conclusions}
\section{Current Status}

Buffalo is currently at version 0.2. It contains two types of aspect:  MethodBoundaryAspect - where various execution points can be intercepted. MethodAroundAspect - where a method can be completely replaced while preserving the option to call back into the original method. MethodAroundAspect targets individual method, whereas MethodBoundaryAspect can be applied on three levels on a .NET application.

Originally MSBuild integration was planned, that when the Buffalo program was installed via the setup executable, it will modify the relevant .NET configuration files to hook BuffaloAOP.exe into MSBuild. That would trigger the weaving process from within the Visual Studio IDE when the solution is compiled. It was later found that in the latest version of Visual Studio 2012, Microsoft has dropped support for the setup project type. As a work around, instruction is provided in appendix A to manually hook into MSBuild.

\section{Future Work}

There are couple areas where Buffalo can be improved upon. Usability wise, currently there is no automatic setup program that installs Buffalo onto user’s computer. There is also no automatic integration into MS-Build System. Some manual steps are still needed in order to provide a more seamless experience.

Functionality wise, when Buffalo performs post compilation weaving it starts fresh each time; it would be interesting to see how incremental weaving can be done here. Another useful functionality is to be able to apply aspects by matching a set of methods.

\section{Lessons Learned}

A framework such as Buffalo mitigates the problem of cross-cutting concerns. Still, to efficiently tackle the root of the problem, compiler vendors have to actively embrace the AOP concept and provide native support in their languages.

Developers also have to understand such problems and what solutions are available to better educate themselves.

Only when the concept is widely understood and supported by both developers and vendors can there hope to begin alleviating such problems.
