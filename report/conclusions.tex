\chapter{Conclusions}
\section{Current Status}

Buffalo is currently at version 0.2. It contains two types of aspect: MethodBoundaryAspect, where various execution points can be intercepted. And MethodAroundAspect, where a method can be completely replaced while preserving the option to call back into the original method. Currently aspects can be applied on three levels on a .NET application.

\section{Future Work}

There are couple areas where Buffalo can be improved upon. Usability wise, currently there is no automatic setup program that install Buffalo onto user’s computer and integrate into MS-Build System. Some manual steps are still needed in order to provide a more seamless experience.

Functionality wise, when Buffalo performs post compilation weaving it starts fresh each time, it would be interesting to see how incremental weaving can be used here. Another useful functionality is to be able to apply aspects by matching a set of methods.

\section{Lessons Learned}

While framework such as Buffalo mitigates the problem of cross-cutting concerns. Still, to efficiently tackle the root of the problem, compiler vendors have to actively embrace the AOP concept and provide native support in the their languages. 

Developers also have to understand the existent of such problems and solutions available to better educate themselves.

Only when the concept is widely understood and supported by both developers and vendors can there hope to begin alleviating such problems.
