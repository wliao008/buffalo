\chapter{Conclusions}
\section{Current Status}

Buffalo currently contains two types of aspect:  MethodBoundaryAspect - where various execution points can be intercepted - and MethodAroundAspect - where a method can be wrapped around or completely replaced while preserving the option to call back into the original method. MethodAroundAspect targets the individual method, whereas MethodBoundaryAspect can be applied on three levels on a .NET application.

Originally MS-Build integration was planned. When the Buffalo program was installed via the setup executable, it was intended for it to modify the relevant .NET configuration files to hook BuffaloAOP.exe into MS-Build. That would trigger the weaving process from within the Visual Studio IDE when the solution is compiled. It was later found that Microsoft has dropped support for the setup project type. As a work around, instruction on how to hook into MS-Build manually is provided in Appendix A.

Originally, MethodAroundAspect was intended to be used similar to MethodBoundaryAspect, where it could be applied on three levels. However, it was later determined that since the CIL instruction of the actual aspect will also have to be modified, it really does not make sense for it to be applied to more than one method as it would introduce conflicting changes.

MethodAroundAspect improvement remains to be determined.

\section{Future Work}

There are several areas where Buffalo can be improved. Usability-wise, there is currently no automatic setup program that installs Buffalo onto users' computer. There is also no automatic integration into MS-Build System. Some manual steps are still needed in order to provide a more seamless experience.

Functionality-wise, when Buffalo performs Post Compilation Weaving it starts fresh each time; it would be interesting to see how incremental weaving can be done here. Another useful functionality is to be able to apply aspects by matching a set of methods.

\section{Lessons Learned}

A framework such as Buffalo mitigates the problem of cross-cutting concerns. Still, to tackle the root of the problem efficiently, compiler vendors must actively embrace the AOP concept and provide native support in their programming languages.

Developers also have to understand such problems and what solutions are available to better educate themselves.

Only when the concept is widely understood and supported by both developers and vendors can there hope to begin alleviating such problems.
