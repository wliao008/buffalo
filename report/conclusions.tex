\chapter{Conclusions}
\section{Current Status}

Buffalo is currently at version 0.2. It contains two types of aspect: MethodBoundaryAspect, where various execution points can be intercepted. And MethodAroundAspect, where a method can be completely replaced while preserving the option to call back into the original method.

\section{Future Work}

There are couple areas where Buffalo can be improved upon. Usability wise, currently there is no automatic setup program that install Buffalo onto user’s computer and integrate into MS-Build System. Some manual steps are needed in order to provide a more seamless experience.

Functionality wise, another idea that I am interested in is to inject new features into the assembly. Similar to the Introduction feature of AspectJ. 

Incremental compilation is another area I hope to get more familiar with. When Buffalo does post compilation weaving it starts fresh each time. I am interested if incremental weaving can be used here and if so, how.

\section{Lessons Learned}

While working on Buffalo I learned a lot about designing API in general. I tried to follow the best practice on what to expose and what to hide from the user, making the API usage more simplified and consistent. I also have a better understanding of the overall Aspect Oriented Programming concepts and various implementations, and how I can make use of it to eliminate some of the cross-cutting concerns in my applications.
