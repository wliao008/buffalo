\begin{lstlisting}[caption={../buffalo/MethodAroundAspect.cs}, label=../buffalo/MethodAroundAspect.cs, frame=tb, basicstyle=\scriptsize]using Buffalo.Interfaces;

namespace Buffalo
{
    public abstract class MethodAroundAspect : AspectBase, IMethodAroundAspect
    {
        public virtual object Invoke(MethodArgs args) { return null; }
    }
}
\end{lstlisting}

\begin{lstlisting}[caption={../buffalo/MethodArgs.cs}, label=../buffalo/MethodArgs.cs, frame=tb, basicstyle=\scriptsize]using System;
using System.Collections.Generic;

namespace Buffalo
{
    public sealed class MethodArgs
    {
        private string name;
        private string fullName;
        private string returnTypeStr;
        private string parameterStr;
        private List<Parameter> parameters;
        private object[] parameterArray;
        private Exception exception;
        private Object instance;

        public MethodArgs()
        {
            this.parameters = new List<Parameter>();
        }

        public string Name
        {
            get
            {
                return this.name;
            }
        }

        public string FullName
        {
            get
            {
                return this.fullName;
            }
        }

        public Type ReturnType
        {
            get
            {
                return Type.GetType(this.returnTypeStr);
            }
        }

        public List<Parameter> Parameters
        {
            get { return this.parameters; }
        }

        public Exception Exception
        {
            get { return this.exception; }
        }

        public object Instance
        {
            get { return this.instance; }
        }

        public object[] ParameterArray
        {
            get { return this.parameterArray; }
        }

        public void SetProperties(string name,
            string fullname,
            string returnTypeStr,
            string parameterStr,
            object[] parameterArray,
            object instance = null)
        {
            this.name = name;
            this.fullName = fullname;
            this.returnTypeStr = returnTypeStr;
            this.parameterStr = parameterStr;
            this.parameterArray = parameterArray;
            this.instance = instance;

            var splits = this.parameterStr.Split(new char[] { '|' }, StringSplitOptions.RemoveEmptyEntries);
            foreach (var split in splits)
            {
                var p = split.Split(new char[] { ':' }, StringSplitOptions.RemoveEmptyEntries);
                this.parameters.Add(new Parameter { Name = p[0], Type = Type.GetType(p[1]) });
            }

            for (int i = 0; i < this.Parameters.Count; ++i)
            {
                this.Parameters[i].Value = this.parameterArray[i];
            }
        }

        public void SetException(Exception exception)
        {
            this.exception = exception;
        }

        public object Proceed()
        {
            return null;
        }
    }
}
\end{lstlisting}

\begin{lstlisting}[caption={../buffalo/Aspect.cs}, label=../buffalo/Aspect.cs, frame=tb, basicstyle=\scriptsize]using System;
using Buffalo.Common;
using Mono.Cecil;

namespace Buffalo
{
    internal class Aspect : IAspect
    {
        public Aspect()
        {
            this.AssemblyLevelStatus = Enums.Status.NotApplied;
        }

        public string Name { get; set; }

        public Enums.Status AssemblyLevelStatus { get; set; }

        public TypeDefinition TypeDefinition { get; set; }

        [Obsolete]
        public System.Type Type { get; set; }

        public Buffalo.Common.Enums.BuffaloAspect BuffaloAspect { get; set; }

        public override string ToString()
        {
            return this.Name;
        }
    }
}
\end{lstlisting}

\begin{lstlisting}[caption={../buffalo/MethodBoundaryAspect.cs}, label=../buffalo/MethodBoundaryAspect.cs, frame=tb, basicstyle=\scriptsize]using Buffalo.Interfaces;

namespace Buffalo
{
    public abstract class MethodBoundaryAspect : AspectBase, IMethodBoundaryAspect
    {
        public virtual void OnBefore(MethodArgs args) { }

        public virtual void OnAfter(MethodArgs args) { }

        public virtual void OnSuccess(MethodArgs args) { }

        public virtual void OnException(MethodArgs args) { }
    }
}
\end{lstlisting}

\begin{lstlisting}[caption={../buffalo/AspectBase.cs}, label=../buffalo/AspectBase.cs, frame=tb, basicstyle=\scriptsize]namespace Buffalo
{
    [System.AttributeUsage(System.AttributeTargets.All,
        AllowMultiple = false)]
    public abstract class AspectBase : System.Attribute
    {
        public AspectBase(bool attributeExclude = false)
        {
            this.AttributeExclude = attributeExclude;
        }

        public bool AttributeExclude { get; set; }
    }
}
\end{lstlisting}

\begin{lstlisting}[caption={../buffalo/Weaver.cs}, label=../buffalo/Weaver.cs, frame=tb, basicstyle=\scriptsize]using Buffalo.Common;
using Buffalo.Injectors;
using Buffalo.Interfaces;
using Mono.Cecil;
using Mono.Cecil.Cil;
using System;
using System.Collections.Generic;
//using Mono.Cecil.Rocks;
using System.Collections.Specialized;
using System.IO;
using System.Linq;
using Reflection = System.Reflection;

namespace Buffalo
{
    internal class Weaver
    {
        public Weaver(string assemblyPath)
        {
            ///TODO: Maybe just don't do anything if file not found
            if (!File.Exists(assemblyPath))
                throw new FileNotFoundException();

            AssemblyPath = assemblyPath;
            this.Init();
        }

        static internal string AssemblyPath { get; set; }

        static internal List<Aspect> Aspects { get; set; }

        static internal Dictionary<string, Type> UnderlyingAspectTypes { get; set; }

        internal Dictionary<MethodDefinition, List<Aspect>> EligibleMethods { get; set; }

        internal List<TypeDefinition> TypeDefinitions { get; set; }

        internal AssemblyDefinition AssemblyDefinition { get; set; }

        internal StringCollection NewMethodNames { get; set; }

        internal void Inject(string outPath)
        {
            var injectors = new List<IInjectable>();

            //apply the around aspect if necessary
            var aroundAspectExist = this.EligibleMethods.Values.Any(x => 
                x.Any(y => y.BuffaloAspect == Enums.BuffaloAspect.MethodAroundAspect));
            if (aroundAspectExist)
                injectors.Add(new MethodAroundInjector());

            //apply the boundary aspect if necessary
            var boundaryAspectExist = this.EligibleMethods.Values.Any(x =>
                x.Any(y => y.BuffaloAspect == Enums.BuffaloAspect.MethodBoundaryAspect));
            if (boundaryAspectExist)
                injectors.Add(new MethodBoundaryInjector());

            //inject
            injectors.ForEach(x => x.Inject(this.AssemblyDefinition, this.EligibleMethods));

            //write out the modified assembly
            this.AssemblyDefinition.Write(outPath);
            Console.WriteLine("DONE");
        }

        private void Init()
        {
            //initialize the variables
            Aspects = new List<Aspect>();
            NewMethodNames = new StringCollection();
            this.TypeDefinitions = new List<TypeDefinition>();
            this.EligibleMethods = new Dictionary<MethodDefinition, List<Aspect>>();
            //set the resolver in case assembly is in different directory
            var resolver = new DefaultAssemblyResolver();
            resolver.AddSearchDirectory(new FileInfo(AssemblyPath).Directory.FullName);
            var parameters = new ReaderParameters { AssemblyResolver = resolver };
            this.AssemblyDefinition = AssemblyDefinition.ReadAssembly(AssemblyPath, parameters);
            //populate the type definition first
            foreach (var m in this.AssemblyDefinition.Modules)
                m.Types.ToList().ForEach(x => this.TypeDefinitions.Add(x));
            //if aspects are defined in a different assembly?
            var typedefs = this.FindAspectTypeDefinition();
            this.TypeDefinitions = this.TypeDefinitions.Union(typedefs).ToList();
            //extract aspects from the type definitions
            this.TypeDefinitions
                .Where(x => x.BaseType != null)
                .ToList()
                .ForEach(x =>
                {
                    Buffalo.Common.Enums.BuffaloAspect? ba = null;
                    if (x.BaseType.FullName == typeof(MethodBoundaryAspect).FullName)
                        ba = Enums.BuffaloAspect.MethodBoundaryAspect;
                    else if (x.BaseType.FullName == typeof(MethodAroundAspect).FullName)
                        ba = Enums.BuffaloAspect.MethodAroundAspect;
                    if(ba.HasValue)
                        Aspects.Add(new Aspect { Name = x.FullName, TypeDefinition = x, BuffaloAspect = ba.Value });
                });
            Aspects.ForEach(x => x.AssemblyLevelStatus = this.CheckAspectStatus(this.AssemblyDefinition, x));
            //finally, get all the eligible methods for each aspect
            Aspects
                .Where(x => x.AssemblyLevelStatus != Enums.Status.Excluded)
                .ToList()
                .ForEach(x =>
                {
#if DEBUG
                    Console.WriteLine("Aspect {0}: {1}", x.Name, x.AssemblyLevelStatus.ToString());
                    Console.WriteLine("============================================");
#endif
                    this.CheckEligibleMethods(x);
                    Console.WriteLine("");
                });
        }

        private List<TypeDefinition> FindAspectTypeDefinition()
        {
            //look for aspect in this assembly, if aspect is defined in a different
            //assembly, import it here.
            var types = this.AssemblyDefinition.MainModule.Types;
            var tdefs = new List<TypeDefinition>();
            foreach (var ca in this.AssemblyDefinition.CustomAttributes)
            {
                var car = ca.AttributeType.Resolve();
                if (car.BaseType.FullName == typeof(MethodBoundaryAspect).FullName
                    || car.BaseType.FullName == typeof(MethodAroundAspect).FullName)
                {
                    tdefs.Add(car);
                }
            }

            //loop thru the custom attributes of each type, resolve them to find aspects
            foreach (var type in types)
            {
                if (type.CustomAttributes.Count == 0) continue;
                var cas = type.CustomAttributes;
                foreach (var ca in cas)
                {
                    var car = ca.AttributeType.Resolve();
                    if (car.BaseType.FullName == typeof(MethodBoundaryAspect).FullName
                        || car.BaseType.FullName == typeof(MethodAroundAspect).FullName)
                    {
                        if(!tdefs.Contains(car))
                            tdefs.Add(car);
                    }
                }
            }

            return tdefs;
        }

        private void PrintEligibleMethods()
        {
            foreach (var de in this.EligibleMethods)
            {
                Console.WriteLine(de.Key.FullName);
                foreach (var a in de.Value)
                    Console.WriteLine("\t" + a.Name);
            }
        }

        private void CheckEligibleMethods(Aspect aspect)
        {
            foreach (var t in this.TypeDefinitions.Where(x => !x.Name.Equals("<Module>")
                && (x.BaseType == null || (x.BaseType.FullName != typeof(MethodBoundaryAspect).FullName
                    && x.BaseType.FullName != typeof(MethodAroundAspect).FullName))))
            {
                var status = this.CheckAspectStatus(t, aspect);
#if DEBUG
                Console.WriteLine("\t{0}: {1}", t.Name, status.ToString());
#endif
                if (status == Enums.Status.Excluded)
                    continue;

                var mths = this.GetMethodDefinitions(t, status, aspect);
                mths.ForEach(x =>
                {
                    if (!this.EligibleMethods.ContainsKey(x))
                    {
                        this.EligibleMethods.Add(x, new List<Aspect>() { aspect });
                    }
                    else
                    {
                        var aspects = this.EligibleMethods[x];
                        aspects.Add(aspect);
                    }
                });
            }
        }

        private List<MethodDefinition> GetMethodDefinitions(TypeDefinition typeDef, Enums.Status typeStatus, Aspect aspect)
        {
            var list = new List<MethodDefinition>();
            foreach (var method in typeDef.Methods)
            {
                var status = this.CheckAspectStatus(method, aspect);
                if (status == Enums.Status.Applied)
                {
                    list.Add(method);
                }
                else
                {
                    if (typeStatus == Enums.Status.Applied && status != Enums.Status.Excluded)
                    {
                        status = Enums.Status.Applied;
                        list.Add(method);
                    }
                }
#if DEBUG
                Console.WriteLine("\t\t{0}: {1}", method.Name, status.ToString());
#endif
            }

            return list;
        }

        /// <summary>
        /// A TypeDefinition and MethodDefinition both implement the
        /// ICustomAttributeProvider interface, so it can be used here
        /// to determined if a method is marked as exclude or not.
        /// </summary>
        private Enums.Status CheckAspectStatus(ICustomAttributeProvider def, Aspect aspect)
        {
            Enums.Status status = aspect.AssemblyLevelStatus;

            bool attrFound = false;
            for (int i = 0; i < def.CustomAttributes.Count; ++i)
            {
                if (def.CustomAttributes[i].AttributeType.FullName.Equals("System.Runtime.CompilerServices.CompilerGeneratedAttribute"))
                {
                    status = Enums.Status.Excluded;
                    break;
                }

                if (aspect.TypeDefinition != null
                    && (aspect.BuffaloAspect == Enums.BuffaloAspect.MethodBoundaryAspect
                    || aspect.BuffaloAspect == Enums.BuffaloAspect.MethodAroundAspect))
                {
                    attrFound = true;
                    if (def.CustomAttributes[i].Properties.Count == 0)
                    {
                        status = Enums.Status.Applied;
                    }
                    else
                    {
                        var exclude = def.CustomAttributes[i].Properties.FirstOrDefault(x => x.Name == "AttributeExclude");
                        if (exclude.Argument.Value != null && (bool)exclude.Argument.Value == true)
                        {
                            status = Enums.Status.Excluded;
                            def.CustomAttributes.RemoveAt(i);
                        }
                    }
                }
            }

            if (!attrFound && aspect.AssemblyLevelStatus == Enums.Status.Applied)
            {
                //this aspect is applied on the assembly level and
                //as a result the type and method might not have the
                //attributed annotated, this is to programmatically add
                //in the annotation so IL can be generated correctly.
                var ctor = aspect.TypeDefinition.Methods.First(x => x.IsConstructor);
                var ctoref = this.AssemblyDefinition.MainModule.Import(ctor);
                def.CustomAttributes.Add(new CustomAttribute(ctoref));
            }

            return status;
        }
    }
}\end{lstlisting}

\begin{lstlisting}[caption={../buffalo/Parameter.cs}, label=../buffalo/Parameter.cs, frame=tb, basicstyle=\scriptsize]using System;

namespace Buffalo
{
    public sealed class Parameter
    {
        public Type Type { get; set; }

        public string Name { get; set; }

        public object Value { get; set; }
    }
}
\end{lstlisting}

\begin{lstlisting}[caption={../buffalo/Injectors/MethodAroundInjector.cs}, label=../buffalo/Injectors/MethodAroundInjector.cs, frame=tb, basicstyle=\scriptsize]using Buffalo.Extensions;
using Buffalo.Interfaces;
using Mono.Cecil;
using Mono.Cecil.Cil;
using System;
using System.Collections.Generic;
using System.Collections.Specialized;
using System.Linq;
//using Mono.Cecil.Rocks;

namespace Buffalo.Injectors
{
    internal class MethodAroundInjector : IInjectable
    {
        AssemblyDefinition AssemblyDefinition;
        Dictionary<MethodDefinition, List<Aspect>> EligibleMethods;

        public void Inject(Mono.Cecil.AssemblyDefinition assemblyDefinition, 
            Dictionary<Mono.Cecil.MethodDefinition, List<Aspect>> eligibleMethods)
        {
            /* The around aspect is to intercept all calls to the target method, and
             * replace those calls with a completely different method. While preserving
             * the option to call back to the target method when necessary.
             */
            this.AssemblyDefinition = assemblyDefinition;
            this.EligibleMethods = eligibleMethods;
            var NewMethodNames = new StringCollection();
            var eligibleAroundMethods = this.EligibleMethods
                .Where(x => x.Value.Any(y => y.Type.BaseType == typeof(MethodAroundAspect)));
            foreach (var d in eligibleAroundMethods)
            {
                var method = d.Key;
                var aspects = d.Value;
                var il = method.Body.GetILProcessor();
                var methodType = method.DeclaringType;
                var maInstructions = new List<Instruction>();
                var aspectVarInstructions = new List<Instruction>();
                var aroundInstructions = new List<Instruction>();

                foreach (var aspect in aspects.Where(x => x.Type.BaseType == typeof(MethodAroundAspect)))
                {
                    var varTicks = System.DateTime.Now.Ticks;

                    //create a replacement for the annotated function
                    var methodName = string.Format("{0}{1}", method.Name, varTicks);
                    MethodDefinition newmethod =
                        new MethodDefinition(methodName, method.Attributes, method.ReturnType);
                    methodType.Methods.Add(newmethod);
                    NewMethodNames.Add(methodName);
                    //newmethod.Body.SimplifyMacros();
                    newmethod.Body.InitLocals = true;

                    //create aspect variable
                    var varAspectName = "asp" + varTicks;
                    var varAspect = new VariableDefinition(varAspectName, aspect.TypeDefinition);
                    newmethod.Body.Variables.Add(varAspect);
                    var varAspectIdx = newmethod.Body.Variables.Count - 1;
                    var constructorInfo = aspect.Type.GetConstructor(new Type[] { });
                    MethodReference myClassConstructor =
                        this.AssemblyDefinition.MainModule.Import(constructorInfo);
                    //store the newly created aspect variable
                    newmethod.Body.Instructions.Add(Instruction.Create(OpCodes.Newobj, myClassConstructor));
                    newmethod.Body.Instructions.Add(Instruction.Create(OpCodes.Stloc, varAspect));
                    //copy all the paramters
                    method.Parameters.ToList().ForEach(x =>
                        newmethod.Parameters.Add(new ParameterDefinition(x.Name, x.Attributes, x.ParameterType)));
                    //create a MethodArgs
                    var var = newmethod.AddMethodArgsVariable(this.AssemblyDefinition);

                    #region Calling MethodArgs.Invoke
                    newmethod.Body.Instructions.Add(Instruction.Create(OpCodes.Ldloc, varAspect));
                    newmethod.Body.Instructions.Add(Instruction.Create(OpCodes.Ldloc, var.Var));
                    var aspectInvoke = aspect.Type.GetMethod("Invoke");
                    var aspectInvokeRef = 
                        this.AssemblyDefinition.MainModule.Import(aspectInvoke, newmethod);
                    newmethod.Body.Instructions.Add(Instruction.Create(OpCodes.Callvirt, aspectInvokeRef));
                    #endregion

                    #region Handling return value
                    if (!newmethod.ReturnType.FullName.Equals("System.Void"))
                    {
                        //create an object variable to hold the return value
                        var varObj = new VariableDefinition("obj" + varTicks,
                            this.AssemblyDefinition.MainModule.Import(typeof(object)));
                        newmethod.Body.Variables.Add(varObj);
                        newmethod.Body.Instructions.Add(
                            Instruction.Create(OpCodes.Stloc, varObj));
                        newmethod.Body.Instructions.Add(
                            Instruction.Create(OpCodes.Ldloc, varObj));
                        newmethod.Body.Instructions.Add(
                            Instruction.Create(OpCodes.Unbox_Any, newmethod.ReturnType));
                    }
                    else
                    {
                        //pop the return value since it's not used?
                        newmethod.Body.Instructions.Add(
                            Instruction.Create(OpCodes.Pop));
                    }
                    #endregion

                    #region Handling Proceed()
                    var invoke = aspect.TypeDefinition.Methods.FirstOrDefault(
                        x => x.FullName.Contains("::Invoke(Buffalo.MethodArgs)"));
                    bool found = false;
                    int instIdx = 0;
                    for (; instIdx < invoke.Body.Instructions.Count; ++instIdx)
                    {
                        if (invoke.Body.Instructions[instIdx].ToString()
                            .Contains("System.Object Buffalo.MethodArgs::Proceed"))
                        {
                            found = true;
                            break;
                        }
                    }

                    if (found)
                    {
                        var invokeInstructions = new List<Instruction>();
                        //var ins = Instruction.Create(OpCodes.Call, method);
                        //invoke.Body.Instructions[instIdx] = ins;
                        invoke.Body.Instructions.RemoveAt(instIdx);
                        var startIdx = instIdx;

                        //create a var to hold the original method type instance
                        var instance = new VariableDefinition("instance" + DateTime.Now.Ticks,
                            this.AssemblyDefinition.MainModule.Import(typeof(object)));
                        invoke.Body.Variables.Add(instance);
                        invoke.Body.InitLocals = true;

                        //get the instance obj from MethodArgs
                        var getInstance = typeof(MethodArgs).GetMethod("get_Instance");
                        var getInstanceRef = this.AssemblyDefinition.MainModule.Import(getInstance);
                        invokeInstructions.Add(Instruction.Create(OpCodes.Callvirt, getInstanceRef));
                        invokeInstructions.Add(Instruction.Create(OpCodes.Stloc, instance));

                        //create object array to hold ParameterArray
                        var objType = this.AssemblyDefinition.MainModule.Import(typeof(object));
                        var objArray = new ArrayType(objType);
                        var varArray = new VariableDefinition("va" + DateTime.Now.Ticks,
                            (TypeReference)objArray);
                        invoke.Body.Variables.Add(varArray);
                        //invokeInstructions.Add(Instruction.Create(OpCodes.Ldc_I4, method.Parameters.Count));
                        //invokeInstructions.Add(Instruction.Create(OpCodes.Newarr, objType));
                        var getParameterArray = typeof(MethodArgs).GetMethod("get_ParameterArray");
                        var getParameterArrayRef = this.AssemblyDefinition.MainModule.Import(getParameterArray);
                        invokeInstructions.Add(Instruction.Create(OpCodes.Ldarg_1));
                        invokeInstructions.Add(Instruction.Create(OpCodes.Callvirt, getParameterArrayRef));
                        invokeInstructions.Add(Instruction.Create(OpCodes.Stloc, varArray));

                        //modify the Invoke() instruction to make a call to the original method
                        invokeInstructions.Add(Instruction.Create(OpCodes.Ldloc, instance));
                        invokeInstructions.Add(Instruction.Create(OpCodes.Unbox_Any, method.DeclaringType));
                        if (method.Parameters.Count > 0)
                        {
                            for (int i = 0; i < method.Parameters.Count; ++i)
                            {
                                invokeInstructions.Add(Instruction.Create(OpCodes.Ldloc, varArray));
                                invokeInstructions.Add(il.Create(OpCodes.Ldc_I4, i));
                                invokeInstructions.Add(il.Create(OpCodes.Ldelem_Ref));
                                invokeInstructions.Add(Instruction.Create(OpCodes.Unbox_Any, method.Parameters[i].ParameterType));
                            }
                        }

                        //make the call
                        invokeInstructions.Add(Instruction.Create(OpCodes.Callvirt, method));

                        #region Handling return value
                        if (!method.ReturnType.FullName.Equals("System.Void"))
                        {
                            //create an object variable to hold the return value
                            var varObj = new VariableDefinition("obj" + DateTime.Now.Ticks,
                                this.AssemblyDefinition.MainModule.Import(typeof(object)));
                            invoke.Body.Variables.Add(varObj);
                            invokeInstructions.Add(
                                Instruction.Create(OpCodes.Box, method.ReturnType));
                        }
                        else
                        {
                            //method is suppose to return void, but since
                            //previously it calls Proceed() which returns object type,
                            //we need to handle that.
                            invoke.Body.Instructions[instIdx] = Instruction.Create(OpCodes.Nop);
                        }
                        #endregion

                        //write out the instruction
                        invokeInstructions.ForEach(
                            x => invoke.Body.Instructions.Insert(startIdx++, x));
                    }
                    #endregion

                    #region Modify all calls from origin to the generated method
                    foreach (var type in this.AssemblyDefinition.MainModule.Types
                        .Where(x => x.BaseType == null || !x.BaseType.FullName.Equals("Buffalo.MethodAroundAspect")))
                    {
                        var methods = type.Methods.Where(x => !NewMethodNames.Contains(x.Name));
                        foreach (var m in methods)
                        {
                            for (int j = 0; j < m.Body.Instructions.Count; ++j)
                            {
                                if (m.Body.Instructions[j].ToString().Contains(method.FullName))
                                {
                                    m.Body.Instructions[j].Operand = newmethod;
                                    //MethodArgs.Invoke returns an object, need to unbox it here to the original type
                                    //However, unboxing is needed only if the original method has a return type
                                    //other than void
                                    if (!newmethod.ReturnType.FullName.Equals("System.Void"))
                                    {
                                        //var unbox = Instruction.Create(OpCodes.Unbox_Any, newmethod.ReturnType);
                                        //var il2 = m.Body.GetILProcessor();
                                        //il2.InsertAfter(m.Body.Instructions[j], unbox);
                                    }
                                }
                            }
                        }
                    }
                    #endregion

                    newmethod.Body.Instructions.Add(Instruction.Create(OpCodes.Ret));
                    //newmethod.Body.OptimizeMacros();
                }
            }
        }
    }
}
\end{lstlisting}

\begin{lstlisting}[caption={../buffalo/Injectors/MethodBoundaryInjector.cs}, label=../buffalo/Injectors/MethodBoundaryInjector.cs, frame=tb, basicstyle=\scriptsize]using Buffalo.Common;
using Buffalo.Extensions;
using Buffalo.Interfaces;
using Mono.Cecil;
using Mono.Cecil.Cil;
using System;
using System.Collections.Generic;
using System.Linq;
//using Mono.Cecil.Rocks;

namespace Buffalo.Injectors
{
    internal class MethodBoundaryInjector : IInjectable
    {
        AssemblyDefinition AssemblyDefinition;
        Dictionary<MethodDefinition, List<Aspect>> EligibleMethods;

        public void Inject(AssemblyDefinition assemblyDefinition, Dictionary<MethodDefinition, List<Aspect>> eligibleMethods)
        {
            this.AssemblyDefinition = assemblyDefinition;
            this.EligibleMethods = eligibleMethods;

            var ems = this.EligibleMethods.ToList();
            var eligibleBoundaryMethods = ems.Where(x => x.Value.Any(y =>
                y.BuffaloAspect == Enums.BuffaloAspect.MethodBoundaryAspect));
            foreach (var d in eligibleBoundaryMethods)
            {
                var method = d.Key;
                if (method.Body == null)
                {
#if DEBUG
                    Console.WriteLine(string.Format("{0} has empty body, skipping", method.FullName));
#endif
                    continue;
                }
                var aspects = d.Value;
                var il = method.Body.GetILProcessor();
                var voidType = method.ReturnType.FullName.Equals("System.Void");
                var ret = il.Create(OpCodes.Ret);

                var maInstructions = new List<Instruction>();
                var aspectVarInstructions = new List<Instruction>();
                var beforeInstructions = new List<Instruction>();
                var successInstructions = new List<Instruction>();
                var exceptionInstructions = new List<Instruction>();
                var afterInstructions = new List<Instruction>();

                //method.Body.SimplifyMacros();
                #region Method detail
                //create a MethodArgs
                var var = method.AddMethodArgsVariable(this.AssemblyDefinition);
                #endregion

                #region Before, Success, Exception, After
                var varExpTypeRef = this.AssemblyDefinition.MainModule.Import(typeof(System.Exception));
                for (int i = 0; i < aspects.Count; ++i)
                {
                    #region Create an aspect variable
                    var varAspectName = "asp" + System.DateTime.Now.Ticks;
                    var varAspect = new VariableDefinition(varAspectName, aspects[i].TypeDefinition);
                    method.Body.Variables.Add(varAspect);
                    var varAspectIdx = method.Body.Variables.Count - 1;
                    //call ctor
                    var ctor = aspects[i].TypeDefinition.Methods.First(x => x.IsConstructor);
                    aspectVarInstructions.Add(Instruction.Create(OpCodes.Newobj, ctor));
                    aspectVarInstructions.Add(Instruction.Create(OpCodes.Stloc, varAspect));
                    #endregion

                    #region Before, success, exception
                    var before = method.FindMethodReference(aspects[i], Enums.AspectType.OnBefore);
                    if (before != null)
                    {

                        beforeInstructions.Add(Instruction.Create(OpCodes.Ldloc, varAspect));
                        beforeInstructions.Add(Instruction.Create(OpCodes.Ldloc, var.Var));
                        var aspectBefore = aspects[i].TypeDefinition.Methods.FirstOrDefault(
                            x => x.Name == Enums.AspectType.OnBefore.ToString());
                        beforeInstructions.Add(Instruction.Create(OpCodes.Callvirt, aspectBefore));
                    }

                    var success = method.FindMethodReference(aspects[i], Enums.AspectType.OnSuccess);
                    if (success != null)
                    {
                        successInstructions.Add(Instruction.Create(OpCodes.Ldloc, varAspect));
                        successInstructions.Add(Instruction.Create(OpCodes.Ldloc, var.Var));
                        var aspectSuccess = aspects[i].TypeDefinition.Methods.FirstOrDefault(
                            x => x.Name == Enums.AspectType.OnSuccess.ToString());
                        successInstructions.Add(Instruction.Create(OpCodes.Callvirt, aspectSuccess));
                    }

                    var exception = method.FindMethodReference(aspects[i], Enums.AspectType.OnException);
                    if (exception != null)
                    {
                        var expName = "exp" + System.DateTime.Now.Ticks;
                        var varExp = new VariableDefinition(expName, varExpTypeRef);
                        method.Body.Variables.Add(varExp);
                        exceptionInstructions.Add(Instruction.Create(OpCodes.Stloc, varExp));
                        exceptionInstructions.Add(Instruction.Create(OpCodes.Ldloc, var.Var));
                        exceptionInstructions.Add(Instruction.Create(OpCodes.Ldloc, varExp));
                        var maSetException = typeof(MethodArgs).GetMethod("SetException");
                        var maSetExceptionRef = this.AssemblyDefinition.MainModule.Import(maSetException, method);
                        exceptionInstructions.Add(Instruction.Create(OpCodes.Callvirt, maSetExceptionRef));

                        exceptionInstructions.Add(Instruction.Create(OpCodes.Ldloc, varAspect));
                        exceptionInstructions.Add(Instruction.Create(OpCodes.Ldloc, var.Var));
                        var aspectException = aspects[i].TypeDefinition.Methods.FirstOrDefault(
                            x => x.Name == Enums.AspectType.OnException.ToString());
                        exceptionInstructions.Add(Instruction.Create(OpCodes.Callvirt, aspectException));
                    }

                    var after = method.FindMethodReference(aspects[i], Enums.AspectType.OnAfter);
                    if (after != null)
                    {
                        afterInstructions.Add(Instruction.Create(OpCodes.Ldloc, varAspect));
                        afterInstructions.Add(Instruction.Create(OpCodes.Ldloc, var.Var));
                        var aspectAfter = aspects[i].TypeDefinition.Methods.FirstOrDefault(
                            x => x.Name == Enums.AspectType.OnAfter.ToString());
                        afterInstructions.Add(Instruction.Create(OpCodes.Callvirt, aspectAfter));
                    }
                    #endregion
                }

                int varIdx = var.VarIdx;
                //maInstructions.ForEach(x => method.Body.Instructions.Insert(varIdx++, x));
                aspectVarInstructions.ForEach(x => method.Body.Instructions.Insert(varIdx++, x));

                int beforeIdx = varIdx;
                //perform this only if user overrides Before() in the aspect
                if (beforeInstructions.Count > 0)
                {
                    beforeInstructions.ForEach(x => method.Body.Instructions.Insert(beforeIdx++, x));
                }

                /* the last instruction after success should jump to return, or 3 instruction before
                 * return if return type is not void, or as an optimization, maybe we can even skip
                 * directly to last() - 2?
                 */
                var successRet = voidType ? method.Body.Instructions.Last() :
                    method.Body.Instructions[method.Body.Instructions.Count - 3];
                Instruction successLeave = il.Create(OpCodes.Leave, successRet);
                ///TODO: need to double check the format of the MSIL return
                ///br.s, ldloc, ret when return type is not void, thereby decrement by 3
                int successIdx = voidType ? method.Body.Instructions.Count - 1 : method.Body.Instructions.Count - 3;
                //perform this only if user overrides Success() in the aspect
                successInstructions.Add(successLeave);
                if (successInstructions.Count > 0)
                {
                    successInstructions.ForEach(x => method.Body.Instructions.Insert(successIdx++, x));
                }

                int exceptionIdx = voidType ? method.Body.Instructions.Count - 1 : method.Body.Instructions.Count - 3;
                int exceptionIdxConst = exceptionIdx;
                var exceptionRet = voidType ? method.Body.Instructions.Last() :
                    method.Body.Instructions[method.Body.Instructions.Count - 3];
                Instruction exceptionLeave = il.Create(OpCodes.Leave, exceptionRet);
                //perform this only if user overrides Exception() in the aspect
                if (exceptionInstructions.Count > 0)
                {
                    exceptionInstructions.Add(exceptionLeave);
                    exceptionInstructions.ForEach(x => method.Body.Instructions.Insert(exceptionIdx++, x));
                }

                var afterRet = voidType ? method.Body.Instructions.Last() :
                    method.Body.Instructions[method.Body.Instructions.Count - 3];
                var endfinally = il.Create(OpCodes.Endfinally);
                int afterIdx = voidType ? method.Body.Instructions.Count - 1 : method.Body.Instructions.Count - 3;
                int afterIdxConst = afterIdx;
                //perform this only if user overrides After() in the aspect
                if (afterInstructions.Count > 0)
                {
                    afterInstructions.Add(endfinally);
                    afterInstructions.ForEach(x => method.Body.Instructions.Insert(afterIdx++, x));
                }

                #endregion

                #region Catch..Finally..
                //add the catch block only if user overrides Exception() in the aspect
                if (exceptionInstructions.Count > 0)
                {
                    var catchHandler = new ExceptionHandler(ExceptionHandlerType.Catch)
                    {
                        TryStart = method.Body.Instructions[varIdx],
                        TryEnd = successLeave.Next,
                        HandlerStart = method.Body.Instructions[exceptionIdxConst],
                        HandlerEnd = exceptionLeave.Next,
                        CatchType = varExpTypeRef,
                    };
                    method.Body.ExceptionHandlers.Add(catchHandler);
                }

                //add the finally block only if user overrides After() in the aspect
                if (afterInstructions.Count > 0)
                {
                    var finallyHandler = new ExceptionHandler(ExceptionHandlerType.Finally)
                    {
                        TryStart = method.Body.Instructions[varIdx],
                        TryEnd = method.Body.Instructions[afterIdxConst],
                        HandlerStart = method.Body.Instructions[afterIdxConst],
                        HandlerEnd = afterRet,
                        CatchType = null,
                    };
                    method.Body.ExceptionHandlers.Add(finallyHandler);
                }
                #endregion
                //method.Body.OptimizeMacros();
            }
        }
    }
}
\end{lstlisting}

\begin{lstlisting}[caption={../buffalo/Properties/AssemblyInfo.cs}, label=../buffalo/Properties/AssemblyInfo.cs, frame=tb, basicstyle=\scriptsize]using System.Reflection;
using System.Runtime.CompilerServices;
using System.Runtime.InteropServices;

// General Information about an assembly is controlled through the following 
// set of attributes. Change these attribute values to modify the information
// associated with an assembly.
[assembly: AssemblyTitle("Buffalo")]
[assembly: AssemblyDescription("")]
[assembly: AssemblyConfiguration("")]
[assembly: AssemblyCompany("Microsoft")]
[assembly: AssemblyProduct("Buffalo")]
[assembly: AssemblyCopyright("Copyright © Microsoft 2012")]
[assembly: AssemblyTrademark("")]
[assembly: AssemblyCulture("")]

// Setting ComVisible to false makes the types in this assembly not visible 
// to COM components.  If you need to access a type in this assembly from 
// COM, set the ComVisible attribute to true on that type.
[assembly: ComVisible(false)]

// The following GUID is for the ID of the typelib if this project is exposed to COM
[assembly: Guid("625c6e79-7034-48cf-8689-b9e44f3bf96d")]

// Version information for an assembly consists of the following four values:
//
//      Major Version
//      Minor Version 
//      Build Number
//      Revision
//
// You can specify all the values or you can default the Build and Revision Numbers 
// by using the '*' as shown below:
// [assembly: AssemblyVersion("1.0.*")]
[assembly: AssemblyVersion("0.1.0.0")]
[assembly: AssemblyFileVersion("0.1.0.0")]
[assembly: InternalsVisibleTo("BuffaloAOP")]\end{lstlisting}

\begin{lstlisting}[caption={../buffalo/Common/BeginEndMarker.cs}, label=../buffalo/Common/BeginEndMarker.cs, frame=tb, basicstyle=\scriptsize]using Mono.Cecil.Cil;

namespace Buffalo.Common
{
    internal struct BeginEndMarker
    {
        public int BeginIndex { get; set; }
        public int EndIndex { get; set; }
    }
}
\end{lstlisting}

\begin{lstlisting}[caption={../buffalo/Common/Enums.cs}, label=../buffalo/Common/Enums.cs, frame=tb, basicstyle=\scriptsize]namespace Buffalo.Common
{
    internal class Enums
    {
        internal enum Status
        {
            Applied,
            NotApplied,
            Excluded
        }

        internal enum AspectType
        {
            OnBefore,
            OnAfter,
            OnSuccess,
            OnException,
            Invoke
        }

        internal enum BuffaloAspect
        {
            MethodBoundaryAspect,
            MethodAroundAspect
        }
    }
}
\end{lstlisting}

\begin{lstlisting}[caption={../buffalo/Extensions/Extensions.cs}, label=../buffalo/Extensions/Extensions.cs, frame=tb, basicstyle=\scriptsize]using Buffalo.Common;
using Mono.Cecil;
using Mono.Cecil.Cil;
using System;
using System.Collections.Generic;
using System.Linq;
using System.Text;

namespace Buffalo.Extensions
{
    internal static class Extensions
    {
        internal static MethodReference FindMethodReference(this MethodDefinition method, Aspect aspect, Enums.AspectType name)
        {
            return aspect
                .TypeDefinition
                .Methods
                .FirstOrDefault(x => x.Name.Equals(name.ToString()));
        }

        internal static VariableResult AddMethodArgsVariable(this MethodDefinition method,
            AssemblyDefinition assemblyDef)
        {
            var var = new VariableResult();
            var instructions = new List<Instruction>();
            method.Body.InitLocals = true;
            var il = method.Body.GetILProcessor();
            var isValueType = false;
            
            //create var to hold parameter count
            var pcVar = new VariableDefinition("pc" + DateTime.Now.Ticks,
                assemblyDef.MainModule.Import(typeof(int)));
            method.Body.Variables.Add(pcVar);
            instructions.Add(Instruction.Create(OpCodes.Ldc_I4, method.Parameters.Count));
            instructions.Add(Instruction.Create(OpCodes.Stloc, pcVar));

            //create an object array to hold the parameter values
            var objType = assemblyDef.MainModule.Import(typeof(object));
            var objArray = new ArrayType(objType);
            var varArray = new VariableDefinition("va" + DateTime.Now.Ticks,
                (TypeReference)objArray);
            method.Body.Variables.Add(varArray);
            instructions.Add(Instruction.Create(OpCodes.Ldloc, pcVar));
            instructions.Add(Instruction.Create(OpCodes.Newarr, objType));
            instructions.Add(Instruction.Create(OpCodes.Stloc, varArray));

            #region Handle paramters
            //loop thru the parameters, extract the values and 
            //store them in MethodArgs.ParameterArray
            TypeSpecification typeSpec = null;
            for (int i = 0; i < method.Parameters.Count; ++i )
            {
                var metaType = method.Parameters[i].ParameterType.MetadataType;
                if (metaType == MetadataType.UIntPtr
                    || metaType == MetadataType.FunctionPointer
                    || metaType == MetadataType.IntPtr
                    || metaType == MetadataType.Pointer)
                    continue;

                instructions.Add(Instruction.Create(OpCodes.Ldloc, varArray));
                instructions.Add(Instruction.Create(OpCodes.Ldc_I4, i));
                if (method.IsStatic)
                    instructions.Add(il.Create(OpCodes.Ldarg, i));
                else
                    instructions.Add(il.Create(OpCodes.Ldarg, i + 1));

                isValueType = false;
                var pType = method.Parameters[i].ParameterType;
                if (pType.IsByReference)
                {
                    typeSpec = pType as TypeSpecification;
                    if (typeSpec != null)
                    {
                        switch (typeSpec.ElementType.MetadataType)
                        {
                            case MetadataType.Boolean:
                            case MetadataType.SByte:
                                instructions.Add(Instruction.Create(OpCodes.Ldind_I1)); 
                                isValueType = true;
                                break;
                            case MetadataType.Int16:
                                instructions.Add(Instruction.Create(OpCodes.Ldind_I2)); 
                                isValueType = true;
                                break;
                            case MetadataType.Int32:
                                instructions.Add(Instruction.Create(OpCodes.Ldind_I4)); 
                                isValueType = true;
                                break;
                            case MetadataType.Int64:
                            case MetadataType.UInt64:
                                instructions.Add(Instruction.Create(OpCodes.Ldind_I8)); 
                                isValueType = true;
                                break;
                            case MetadataType.Byte:
                                instructions.Add(Instruction.Create(OpCodes.Ldind_U1)); 
                                isValueType = true;
                                break;
                            case MetadataType.UInt16:
                                instructions.Add(Instruction.Create(OpCodes.Ldind_U2)); 
                                isValueType = true;
                                break;
                            case MetadataType.UInt32:
                                instructions.Add(Instruction.Create(OpCodes.Ldind_U4)); 
                                isValueType = true;
                                break;
                            case MetadataType.Single:
                                instructions.Add(Instruction.Create(OpCodes.Ldind_R4));
                                isValueType = true;
                                break;
                            case MetadataType.Double:
                                instructions.Add(Instruction.Create(OpCodes.Ldind_R8));
                                isValueType = true;
                                break;
                            case MetadataType.IntPtr:
                            case MetadataType.UIntPtr:
                                instructions.Add(Instruction.Create(OpCodes.Ldind_I));
                                isValueType = true;
                                break;
                            default:
                                if (typeSpec.ElementType.IsValueType)
                                {
                                    instructions.Add(Instruction.Create(OpCodes.Ldobj, typeSpec.ElementType));
                                    isValueType = true;
                                }
                                else
                                {
                                    instructions.Add(Instruction.Create(OpCodes.Ldind_Ref));
                                    isValueType = false;
                                }
                                break;
                        }
                    }
                }

                if (pType.IsValueType || isValueType)
                {
                    if(isValueType)
                        instructions.Add(Instruction.Create(OpCodes.Box, typeSpec.ElementType));
                    else
                        instructions.Add(Instruction.Create(OpCodes.Box, pType));
                }

                instructions.Add(Instruction.Create(OpCodes.Stelem_Ref));
            }
            #endregion

            StringBuilder sb = new StringBuilder();
            method.Parameters.ToList()
                .ForEach(x =>
                {
                    sb.Append(string.Format("{0}:{1}|", x.Name, x.ParameterType.FullName));
                });

            var maType = typeof(MethodArgs);
            var maName = "ma" + DateTime.Now.Ticks;
            var maSetProperties = maType.GetMethod("SetProperties");
            var varMa = new VariableDefinition(maName, assemblyDef.MainModule.Import(maType));
            method.Body.Variables.Add(varMa);
            var vaMaIdx = method.Body.Variables.Count - 1;
            var maCtr = maType.GetConstructor(new Type[] { });
            MethodReference maCtrRef = assemblyDef.MainModule.Import(maCtr);
            instructions.Add(Instruction.Create(OpCodes.Newobj, maCtrRef));
            instructions.Add(Instruction.Create(OpCodes.Stloc, varMa));
            instructions.Add(Instruction.Create(OpCodes.Ldloc, varMa));
            instructions.Add(Instruction.Create(OpCodes.Ldstr, method.Name));
            instructions.Add(Instruction.Create(OpCodes.Ldstr, method.FullName));
            instructions.Add(Instruction.Create(OpCodes.Ldstr, method.ReturnType.FullName));
            instructions.Add(Instruction.Create(OpCodes.Ldstr, sb.ToString()));
            instructions.Add(Instruction.Create(OpCodes.Ldloc, varArray));
            if (method.IsStatic)
                instructions.Add(Instruction.Create(OpCodes.Ldnull));
            else
                instructions.Add(Instruction.Create(OpCodes.Ldarg_0));

            var maSetPropertiesRef = assemblyDef.MainModule.Import(maSetProperties, method);
            instructions.Add(Instruction.Create(OpCodes.Callvirt, maSetPropertiesRef));

            int idx = 0;
            instructions.ForEach(x => method.Body.Instructions.Insert(idx++, x));

            var.Var = varMa;
            var.ParamArray = varArray;
            var.VarIdx = idx;
            return var;
        }
    }

    internal class VariableResult
    {
        public VariableDefinition Var { get; set; }
        public VariableDefinition ParamArray { get; set; }
        public int VarIdx { get; set; }
    }
}
\end{lstlisting}

\begin{lstlisting}[caption={../buffalo/Interfaces/IMethodAroundAspect.cs}, label=../buffalo/Interfaces/IMethodAroundAspect.cs, frame=tb, basicstyle=\scriptsize]namespace Buffalo.Interfaces
{
    internal interface IMethodAroundAspect : IAspect
    {
        object Invoke(MethodArgs args);
    }
}
\end{lstlisting}

\begin{lstlisting}[caption={../buffalo/Interfaces/IInjectable.cs}, label=../buffalo/Interfaces/IInjectable.cs, frame=tb, basicstyle=\scriptsize]using System.Collections.Generic;
using Mono.Cecil;

namespace Buffalo.Interfaces
{
    internal interface IInjectable
    {
        void Inject(AssemblyDefinition assemblyDefinition, Dictionary<MethodDefinition, List<Aspect>> eligibleMethods);
    }
}
\end{lstlisting}

\begin{lstlisting}[caption={../buffalo/Interfaces/IMethodBoundaryAspect.cs}, label=../buffalo/Interfaces/IMethodBoundaryAspect.cs, frame=tb, basicstyle=\scriptsize]namespace Buffalo.Interfaces
{
    internal interface IMethodBoundaryAspect : IAspect
    {
        void OnBefore(MethodArgs args);
        void OnAfter(MethodArgs args);
        void OnSuccess(MethodArgs args);
        void OnException(MethodArgs args);
    }
}
\end{lstlisting}

\begin{lstlisting}[caption={../buffalo/Interfaces/IAspect.cs}, label=../buffalo/Interfaces/IAspect.cs, frame=tb, basicstyle=\scriptsize]namespace Buffalo
{
    /// <summary>
    /// This is the base aspect interface
    /// </summary>
    internal interface IAspect
    {
    }
}
\end{lstlisting}

\begin{lstlisting}[caption={../../buffalo/BuffaloAOP/Program.cs}, label=../../buffalo/BuffaloAOP/Program.cs, frame=tb, basicstyle=\scriptsize]using System;
using System.Collections.Generic;
using System.Linq;
using System.Text;
using System.Reflection;
using Buffalo;

namespace BuffaloAOP
{
    class Program
    {
        static string path;
        static void Main(string[] args)
        {
            if (args == null || args.Count() == 0)
            {
                Console.WriteLine("USAGE: BuffaloAOP.exe <assembly_path>");
                Environment.Exit(1);
            }

            path = args[0];
            path = @"C:\Users\wliao\Documents\Visual Studio 2012\Projects\Tests\BuffaloTest\BuffaloTest\bin\Debug\Shared.dll";
            string outpath = path.Replace(".exe", "_modified.exe").Replace(".dll", "_modified.dll");
            new Weaver(path).Inject(outpath);
            //Console.Read();
        }
    }
}
\end{lstlisting}

