%%%%%%%%%%%%%%%%%%%%%%%%%%%%%%%%%%%%%%%%%%%%%%%%%%%%%%%%%%%%%%%%%%%%%%%%%%%%%%
%
% PROJECT PROPOSAL  DESCRIPTION:
%   A concise description of the main concepts of the proposed project.
%
% RESEARCH:
%   A list of research activities which led to this project.
%
% EXPERIMENTS:
%   A list of the experiments performed which supported the research.
%
%%%%%%%%%%%%%%%%%%%%%%%%%%%%%%%%%%%%%%%%%%%%%%%%%%%%%%%%%%%%%%%%%%%%%%%%%%%%%%%
% Define a single space environment (copied from doublespace.sty)
% e.g. \begin{singlespace}
%         single-spaced text
%      \end{singlespace}

\documentclass[12pt,american]{article}
\usepackage{fullpage}
\usepackage{bbm}
\usepackage{url}
\usepackage{subfigure}
\usepackage{babel}
\usepackage{times}
\usepackage{graphicx}
\usepackage{amssymb}
\usepackage{lscape}
\usepackage{verbatim}
\usepackage{enumerate}
\usepackage{afterpage}
\usepackage{setspace}


\begin{document}
\thispagestyle{empty} 
\begin{center}
{\em MS Project Proposal}\\
\vspace{.5in}
{\large \bf An Aspect Oriented Programming Framework for C\#}\\
\vspace{.5in}
{\bf Wei Liao}\\
\vfill
\
{\em Committee Chair:} Prof. James E. Heliotis\\
\vspace{0.1in}
{\em Reader: } Prof. Fereydoun Kazemian\\
 \vspace{0.1in}
{\em Observer: } Prof. Matthew Fluet\\
 \vspace{0.1in}
Department of Computer Science\\
B. Thomas Golisano College of Computing and Information Sciences \\
Rochester Institute of Technology \\
Rochester, New York \\ [0.3in]
\vspace{0.5in}
\today{}\\
\end{center}
\vfill

%%%%%%%%%%%%%%%%%%%%%%%%%%%%%%%%%%%%%%%%%%%%%%%%%%%%%%%%%%%%%%%%%%%%%%%%%%%%%%%
%%  Collection of useful abbreviations.
\newcommand{\etc} {\emph{etc.\/}}
\newcommand{\etal}{\emph{et~al.\/}}
\newcommand{\eg}  {\emph{e.g.\/}}
\newcommand{\ie}  {\emph{i.e.\/}}
%%%%%%%%%%%%%%%%%%%%%%%%%%%%%%%%%%%%%%%%%%%%%%%%%%%%%%%%%%%%%%%%%%%%%%%%%%%%%%%


%%%%%%%%%%%%%%%%%%%%%%%%%%%%%%%%%%%%%%%%%%%%%%%%%%%%%%%%%%%%%%%%%%%%%%%%%%%%%%%
% Abstract
\section*{Abstract}
Aspect Oriented Programming (AOP) is a paradigm that let programmers isolate and separate cross-cutting concerns from the basis of their program. The concept is relatively new, so very few languages has native support for this capability, support in toolings such as IDE integration is also rare. In this project we will design and implement a framework that provides AOP functionality for C\# via IL code weaving and integrate it with the Visual Studio IDE.
%%%%%%%%%%%%%%%%%%%%%%%%%%%%%%%%%%%%%%%%%%%%%%%%%%%%%%%%%%%%%%%%%%%%%%%%%%%%%%%
\vfill{}

%%%%%%%%%%%%%%%%%%%%%%%%%%%%%%%%%%%%%%%%%%%%%%%%%%%%%%%%%%%%%%%%%%%%%%%%%%%%%%%
% This is where the main body of the capstone proposal starts
\setcounter{page}{0} 
\newpage{}

\section{Introduction}
This part of the proposal should be a couple of paragraphs that
describe the reason for your proposal and your project/thesis area at
high level.

\section{Background}
This section should be sufficient for the reader to understand the
project area and the relevance of your efforts in the world of
computer science. The description here should be provide the
motivation to the reader that you are exploring a problem area that is
relevant to the CS community.


\section{Related Work}
Describe what work others have already done in this area. You do need
several citations, and this is how you cite a book by
Silberschatz~\cite{Silberschatz05-text} or a paper by
Dumont~\cite{Dumont2007-robots}.

\section{Hypothesis}
Summarize what you think the problem is, and what your hypothesis
is. Here is a small example based on a successful project by Priyanka
Sinha: ''Using one technique for schema matching does not seem
adequate. The hypothesis underlying this sproject is that a holistic
approach to schema matching based on the three techniques described
earlier would do an effective approach to schema matching.''

Additional description to circumscribe the work so that the reader
knows what you plan to do to establish your hypothesis.

\section{Solution Design and Implementation}
Describe how you plan to design and implement a solution. 

You must also describe how you would use your solution to establish the validity of your hypothesis. Explain the measurements you plan to conduct and how these would establish the validity (or invalidity) of your hypothesis.

\section{Roadmap}

Here you need to describe your project plan, with dates and deliverables. 

You must review the CS graduate handbook for details, and yes, make
sure you also address all of the handbook's requirements for a
proposal.

%%%%%%%%%%%%%%%%%%%%%%%%%%%%%%%%%%%%%%%%%%%%%%%%%%%%%%%%%%%%%%%%%%%%%%%%%%%%%%%

%%%%%%%%%%%%%%%%%%%%%%%%%%%%%%%%%%%%%%%%%%%%%%%%%%%%%%%%%%%%%%%%%%%%%%%%%%%%%%%
\bibliographystyle{plain}
% Single space the bibliography to save space.
\singlespacing
\bibliography{Proposal}
%%%%%%%%%%%%%%%%%%%%%%%%%%%%%%%%%%%%%%%%%%%%%%%%%%%%%%%%%%%%%%%%%%%%%%%%%%%%%%%


\end{document}
